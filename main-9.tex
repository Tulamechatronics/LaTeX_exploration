\documentclass[12pt]{article}
\usepackage[margin=2cm,a4paper]{geometry}
\usepackage[T1]{fontenc}
\usepackage{fontspec}
\defaultfontfeatures{Mapping=tex-text}
\XeTeXlinebreaklocale"Khm"
\XeTeXlinebreakskip=0pt plus 1pt minus 1pt

\setmainfont[Scale=0.875, Script=Khmer, AutoFakeBold=2.5, AutoFakeSlant=0.3]{Khmer OS Siemreap}
\setsansfont[Scale=0.875, Script=Khmer, AutoFakeBold=2.5, AutoFakeSlant=0.3]{Khmer OS Muol Light}
\setmonofont[Scale=0.875, Script=Khmer, AutoFakeBold=2.5, AutoFakeSlant=0.3]{Khmer OS}
\newcommand{\kos}{\fontspec[Scale=0.875, Script=Khmer]{Khmer OS System}\selectfont}
\newcommand{\kml}{\fontspec[Scale=0.875, Script=Khmer]{Khmer OS Muol Light}\selectfont}
\newcommand{\en}{\fontspec{Times New Roman}\selectfont}
\setmathrm{Times New Roman}
\usepackage{mathtools}
\usepackage{xcolor}
\usepackage{tcolorbox}
\usepackage{amsmath}
\usepackage{amsthm}
\usepackage{amsfonts}
\usepackage{ascmac}
\usepackage{amssymb}
%\usepackage[upint,smallerops]{newpxmath}
\usepackage{mathabx}
\usepackage{tcolorbox}
\usepackage{fancyhdr}
\usepackage{multicol}
\theoremstyle{definition}
\newtheorem{theorem}{{\kml ទ្រឹស្តីបទ}}
\newtheorem{definition}{\kml និយមន័យ}
\newtheorem{remark}{{\kml សម្គាល់}}
\newtheorem{example}{{\kml ឧទាហរណ៍}}
\newenvironment{prf}{{\kml Solution.}}{\hfill$\blacksquare$}
%\usepackage{enumitem}

%\renewcommand\chaptername{\kml ជំពូក}

\DeclareMathSizes{12pt}{13pt}{11pt}{10pt}
\title{\textcolor{red}{\en Banach Spaces: Some Examples and Applications}}
\author{\textcolor{blue}{វេទិកាដើម្បីពង្រីកចំណេះដឹង}}
\date{\today}
\begin{document}

\maketitle
\Large
ដូចបង្ហាញក្នុង Lecture 1 លំហ $\mathbb{R}^n$ ដែលប្រដាប់ដោយណម
\[\|\mathbf{x}\|:=
\begin{cases}
\left(\displaystyle \sum_{j=1}^n|x_j|^p\right)^{1/p} & ,1\le p<\infty\\
\displaystyle\max_{1\le j\le n}|x_j| & ,p=\infty
\end{cases}
\]
ជាលំហ Banach។ មានន័យថា វាជាលំហណម ហើយគ្រប់ស្វីតកូស៊ីជាស្វីតរួម។

បើយើងតាង $e_1=(1,0,\dots,0),e_2=(0,1,0,\dots,0),\dots,e_n=(0,\dots,0,1)$ ។នោះគ្រប់វុិចទ័រទាំងអស់ក្នុង ${\mathbb{R}}^n$ អាចសរសេរជាជាផលបូកនៃពហុគុណរបស់ $e_1,e_2,...,e_n$ ($x=(x_1,x_2,..,x_n)=x_1e_1+x_2e_2+...+x_ne_n$) នោះ $\{e_1,e_2,\dots,e_n\}$ ជាគោលមួយនៃ $\mathbb{R}^n$។ គេថា
$\mathbb{R}^n$ ជាលំហមានវិមាត្រកំណត់ (finite dimensional space) ដែល $\dim \mathbb{R}^n=n$ ព្រោះវាមានបង្ករឡើងដោយគោលដែលមាន $n$ វុិចទ័រ។។\\ 
ខាងក្រោម យើងនឹងពិនិត្យលំហ Banach ដែលមានវិមាត្រ $\infty$។

គួររំលឹកថា បើយើងមានស្វីតចំនួនពិត $(x_n)_{n=1}^{\infty}=(x_1,x_2,\dots),(y_n)_{n=1}^{\infty}=(y_1,y_2,\dots)$ និងចំនួនស្កាលែមួយ $\alpha\in\mathbb{R}$ យើងអាចកំណត់ផលបូកនិងផលគុណនឹងចំនួនស្កាលែដូចខាងក្រោម៖
\begin{align*}
    (x_n+y_n)_{n=1}^{\infty}&:=(x_1+y_1,x_2+y_2,x_3+y_3,\dots)\\
    \alpha (x_n)_{n=1}^{\infty}&:=(\alpha x_1,\alpha x_2,\alpha x_3,\dots)
\end{align*}
ដូច្នេះ យើងគួរតែអាចកំណត់លំហមួយ ដែលជាសំណំុនៃធាតុដែលមានទម្រង់ជាស្វីត៖
\[\mathbb{R}^{\omega}:=\{\mathbf{x}=(x_1,x_2,x_3,\dots)\mid x_j\in \mathbb{R},1\le j\le n\}
\]
ហើយដោយកំណត់ប្រមាណវិធីបូកនិងគុណនឹងចំនួនស្កាលែដូចខាងលើ នោះ $\mathbb{R}^{\omega}$ ក្លាយជាលំហវុិចទ័រ។ បើយើងតាង $e_1=(1,0,\dots),e_2=(0,1,0,\dots),\dots$ នោះ $\{e_j\}_{j=1}^{\infty}=\{e_1,e_2,\dots\}$ ជាគោលមួយនៃ $\mathbb{R}^{\omega}$។ ដោយ ${\mathbb{R}}^{\omega}$ ត្រូវកាធាតុច្រើនរាប់មិនអស់ដើម្បីបង្ករវា ($x=(x_1,x_2,...) \in {\mathbb{R}}^{\omega}$ នោះ $x=x_1e_1+x_2e_2+....$)  គេថា $\mathbb{R}^{\omega}$  ជាលំហវុិចទ័រមានវិមាត្រអនន្ត  យើងសរសេរ $\dim \mathbb{R}^{\omega}=\infty$។
កិច្ចកាបន្ទាប់ដែលយើងចង់បានគឺធ្វើ $\mathbb{R}^{\omega}$ ជាលំហរ Banach មួយ។ ហើយរបស់ដែលចាំបាច់ដំបូងគឺកំណត់ ណម (សញ្ញាញប្រវែង) មួយនៅក្នុងលំហរនេះ។ យើងអាចសាកល្បងអោយមន័យប្រវែងតាមរបៀបដែលយើងធ្វើខាងលើក្នុងករណី ${\mathhbb{R}}^{n}$ គឺយើងកំណត់ $\|x\|_p={(|x_1|^p+|x_2|^p+...)}^{\frac{1}{p}}$។ បញ្ញាមួយរបស់និយមន័យនេះគីនៅត្រង់ថាសេរី $|x_1|^p+|x2|^p+....$ អាចមិនមែនជាសេរីរួម។ ដែលនេះធ្វើឲ្យផលបូកនេះស្មើអន្តន។ មានវិធីពីរដើម្បីដោះស្រាយបញ្ហានេះ។ វិធីមួយគឺរកវិធីថ្មីដើម្បីកំណត់ប្រវែង ដែលវិធីនេះមានសិក្សានៅក្នុងការកតំលៃប្រហែលរបស់សមីកាឌីផេរ៉ងស្យែលជាដើម និងវិធីមួយទៀតគឺសិក្សាតែវុិចទ័រណាដែលមានប្រវែងតូចជាងអន្តន។ ក្នុងអត្ថបទនេះយើងនិងសិក្សាវិធីទីពីរនេះ។
\section{ លំហរ $ {l}^p(\mathbb{R})$}
ជាជាងកាសិក្សាគ្រប់វុិចទ័ទាំងអស់នៅក្នុង ${\mathbb{R}^{\omega}$ យើងសិក្សាតែវិទ័រណាដែលមានប្រវែងស្មើតូចជាងអន្តរ ហើយយើងតាងសំណុំនែវុិចទ័ទាំងនេះដោយ $l^{p}(\mathbb{R})$ ឬ $l^p$។
\begin{equation}
l^p=l^p(\mathbb{R})=\{x\in \mathbb{R}^{\omega}: \|x\|_p=(|x_1|^p+|x_2|^p+....)^{\frac{1}{p}}  <\infty \}
\end{equation}
យើងនិងស្រាយថា $l^p$ ជាលំហរវុីចទ័រមូយដែលមានណមមានលក្ខណពេញ (complete) ឬអាចនិយាយម៉្យាងទៀតថា $(l^p, \|.\|_p)$ ជាលំហរ Banach។ 
ដំបូងយើងស្រាយថាផលបូករបស់ធាតុក្នុង $l^p$  និងផលគុណស្កាលែររបស់ធាត់ក្នុង $l^p$ ជាមួយចំនួនពិតគឺបានលទ្ធផលនៅក្នុង $l^p$ វិញ។ \\
យក$ x=(x_1,x_2,...), y=(y_1,y_2,...)$ ជាធាតុនៅក្នុង $l^p$ នោះយើងនិងស្រាយថា $x+y=(x_1+y_1, x_2+y_2,...)$ និង $\alpah x=(\alpha x_1, \alpha x_2, ...)$ ក៏នៅក្នុង $l^p$ ដែរ។

ទីមួយយើងសង្កេតថា $|a+b|<2\max{\{|a|,|b|\}} $នោះ $|a+b|^p< (2\max{\{|a|,|b|\}})^p=2^p \max{\{|a|^p,|b|^p\}}<2^p(|a|^p+|b|^p)$ ព្រោះ $\max{\{\alpha,\beta \}}< \alpha+\beta$ ចំពោះគ្រប់ $\alpha ,\beta \geq 0$។ ដូចនេះយើងបាន \begin{equation} \display{\sum_{n=1}^{\infty} |x_n+y_n|^p \leq  \sum_{n=1}^{\infty} (2^p(|x_n|^p+|y_n|^p)=2^p(\sum_{n=1}^{\infty}|x_n|^p+\sum_{n=1}^{\infty} |y_n|^p) } \end{equation} នោះយើងបាន \begin{equation} \|x+y\|_p^p\leq 2^p( \|x\|_p^p+\|y\|^p) < \infty \end{equation} ព្រោះ $\|x\|_p <\infty$ និង $\|y\|_p< \infty$។ ។ 

ទីពីរ $ \|\alpha x\|_p={(\display {\sum_{n=1}^{\infty}|\alpha x_n|^p})}^{\frac{1}{p}}=|\alpha|{(\sum_{n=1}^{\infty} |x_n|^p)}^{\frac{1}{p}}<\infty $ (ក)។   \\
សមីកា (3) និងសមីកា (ក) បង្ហាញថាបើវុិចទ័រ2 មានប្រវែងតិចជាងអន្តន នោះផលបូករបស់វាក៏មានប្រវែងតិចជាងអន្តនដែរ និងមួយទៀតប្រសិនបើវុិចទ័រមួយមានប្រវែងតិចជាងអន្តននោះផលគុណស្តាលែរបស់វាជាមួយចំនួនពិតក៏មានប្រវែងតិចជាងអន្តនដែរ។ \\
\subsection{ $l^p$ ជាលំហរវុិចទ័រ }
តាមនិយមន័យផលបូក និងផលគុណស្តាលែររបស់វុិចទ័រក្នុង $l^p$ យើងងាយនិងស្រាយថា $l^p$ ជាលំហរវុិចទ័រមូួយ ពោលគឺប្រមាណវិធីទាំងនេះផ្ទៀងផ្ទាត់លក្ខណដូចខាងក្រោម៖ 
$X_1,X_2,X_3,X\in l^2$ និង $\alpha,\beta\in \mathbb{R}$ 
\begin{itemize}
        \item [(i)] (លក្ខណៈត្រលប់) $X_1+X_2=X_2+X_1$
        \item [(ii)] (លក្ខណៈផ្តំុ)
        $(X_1+X_2)+X_3=X_1+(X_2+X_3)$
        \item [(iii)] (ធាតុណឺតនៃវិធីបូក) មាន $0=(0,0,...)\in l^p$ ដែលចំពោះគ្រប់ 
        $X\in l^p$
        X+0=X
        \item [(iv)] (ធាតុច្រាស) ចំពោះគ្រប់ $X=(x_1,x_2,..)\in l^p$ មាន $X'$ ដែល
        $X+X'=0$
        គេតាង $X'$ ដោយ $-X=(-x_1,-x_2,...)$
        \item [(v)] (លក្ខណៈផ្តុំស្កាលែ) $\alpha(\beta X)=(\alpha \beta)X$
        \item [(vi)] (ធាតុណឺតស្កាលែ) $\mathbf{1}X=X$
        \item [(vii)] (លក្ខណៈពន្លាត I) $(\alpha+\beta )X=\alpha X+\beta X$ 
        \item [(vii)] (លក្ខណៈពន្លាត II) $\alpha(X_1+X_2)=\alpha X_1+\alpha X_2$ 
    \end{itemize} 
    
    យើងគួរយល់ថាលំហរវុិចទ័រគ្រាន់តែជាសំណុំវត្ថុដែលយើងអាចបូក អាចគុណ រុញទៅមកធ្វើប្រម៉ាណវិធី ហើយលទ្ធភាពធ្វើប្រមាណវិធីទាំងនេះគឺជារបស់ដែលយើងត្រូវកានៅក្នងកាវិភាគ និងស្រាយទ្រឹស្តីបទដែលបង្ហាញអំពីលក្ខណៈរបស់ធាតុនៅក្នុងសំណុំនោះ។ កាដែលយើងស្រមៃថាវុីចទ័រក្នុងដូចជា ${\mathbb{B}}^2 $ប្លង់ជាព្រួញ គឺគ្រាន់តែជាជំនួយកាគិតរបស់យើងតែប៉ុន្មោះ តែកាស្រាយវិភាគទាំងអស់ត្រូវអាស្រ័យលើលក្ខណពីជគណិតរបស់វុិទ័រ ហើយលក្ខណមួយចំនួនក្នុងនោះគឺមាននៅក្នុងបញ្ជីខាងលើនេះ។ ដើម្បីជួយឲ្យយើងងាយនិងទទូលយកថា វត្ថុក្នុង
$l^p$ ជាវុិចទ័រដូចគេឯងដែរ យើងគួរព្យាយាមសួរថាអ្វីទៅកាបូកវុិចទ័រ កាគុណស្កាលែរ កាវាស់ប្រវែង ។ល។ នៅពេលដែលយើងគិតអំពីរបស់ទាំងនេះ យើងនិងឃើញថារឿងសំខាន់កាបូកឬកាគុណជាអ្វីវាមិនសំខាន់ទេ រឿងសំខាន់គឺកាបូក និងកាគុណនោះមានលក្ខណដែលយើងចង់បានដូចដែលយើងរៀបរាប់ក្នុងបញ្ជីខាងលើ។
    \subsection {ណមនៅក្នុង$ l^p$}
    
ជាបន្តទៅនេះយើងនិងស្រាយថា $\|.\|: l^p \to [0,\infty ]$ ជាណមនៅក្នុង  $l^p$។  ដំបូងយើងឃើញថា $\|\alpha x\|_p=|\alpha| \|x\|_p$ (សមីកា (ក)ខាងលើ)ជាលក្ខណផលគុណរបស់ណម។ ដូចនេះយើងចាំបាច់ត្រូវការស្រាយតែពីលក្ខណផ្សែងទៀតរបស់ណម។ 
ទីមួយគឺថាបើវុិចទ័រមួយមានណមស្មើរសូន្យនោះវាជាវុិចទ័រសូន្យ។ តាង $x\in l^p$ ជាវុិចទ័រដែល $|x|_p=0$ នោះបើយើងតាង $x=(x_1,x_2,...)$ នោះយើងបាន $\display{\sum_{n=1}^{\infty}{|x_n|^p =0}}$ នោះយើងបាន $|x_n|=0$ ចំពោះគ្រប់ $n=1,2,3...$។ នោះ $x=(0,0,...)$ ជាវុិចទ័រសូន្យពិតប្រកដមែន។កដមែន។

ដើម្បីស្រាយវិសមភាពត្រីកោណយើងនិងស្រាយវិសមភាពវិសមភាណ Young និងវិសមភាព Holder និងក្រោមមកស្រាយវិសមភាណ Minkowski។

{\kml វិសមភាណ \en Young} គេមាន $x,y, p, q$ ជាចំនួនពិតវិជ្ជាមាន។ បើ  $p, q \geq 1$ និងផ្ទៀងផ្ទាត់ $1/p+1/q=1$ នោះ $xy\leq \frac{x^p}{p}+\frac{y^q}{q}$។ 
\begin{proof} 
យក $\theta \in [0,1]$ ជាចំនួនពិតមួយ។ សិក្សា $f: [0,+\intfty) \to \mathbb{R}$ កំណត់ដោយ $f(t)=t^{\theta} -\theta t -(1-\theta)$ នោះ $f'(t)=\theta(t^{\theta -1}-1)$។ យើងឃើញថា $f'(t) \geq 0$ ចំពោះគ្រប់ $t \in [0,1]$ (ព្រោះ $\theta -1 \leq 0$ ) នោះ$ f$ ជាអនុគមន៍កើនលើ $[0,1]$ នាំអោយ$ f(t)\leq f(1)=0$ ចំពោះគ្រប់ $t\in [0,1]$។ ម៉្យាងទៀត $f'(t) \leq 0$ ចំពោះគ្រប់ $t\qeg 1$ នោះ $f$ ជាអនុគមន៍ចុះលើ $[1,+\infty)$ ដែលធ្វើឲ្យ $f(t)\geq f(1)=0$។ សរុបមក $f(t)\leq 0$ ចំពោះគ្រប់ $t \in [0,+\infty)$។ ដូចនេះយើងបាន $x^{\theta} \leq \theta x +(1-\theta)$ ។ យក $t=A/B$ នោះយើងបាន $(A/B)^{\theta} \leq \theta A/B + (1-\theta)$ ឬ $A^{\theta}B^{1-\theta} \leq \theta A^{\theta}+(1-\theta)B^{1-\theta}$។  យក  $\theta =1/p$ នោះ $1-\theta=1/q$  និង $A= x^p, B=y^q$ យើងទទួលបាន $xy\leq \frac{x^p}{p}+\frac{y^q}{q}$។  
\end{proof}
{\kml វិសមភាព Holder}៖ គេមាន $x_1, x_2,...,x_n$ និង $y_1,y_2,...,y_n$ ជាចំនួនពិតមិនអវិជ្ជាមាន។ បើ  $p, q \geq 1$ និងផ្ទៀងផ្ទាត់ $1/p+1/q=1$ នោះ \\ $  (x_1y_1+...+x_ny_n)\leq {(x_1^p+...+x_n^p)}^{1/p}{(y_1^q+...+y_n^q)}^{1/q}$។
\begin{proof}
បើ $x_1^p+...+x_n^p=0$ ឬ $y_1^q+...+y_n^q=0$ នោះ $x_1=...=x_n=0$ ឬ $y_1=...=y_n=0$ ដែលធ្វើឲ្យ $x_1y_1+...+x_ny_n=0$។ ក្នុងករណីនេះវិសមភាព Holder គឺ $0\leq 0$ ដែលជាវិសមភាពពិត។ ដូចនេះយើងចាំបាច់ស្រាយក្នុងករណី $x^p=x_1^p+...+x_n^p  \neq 0$ និង $y^p=y_1^p+...+y_n^p \neq 0$។ ក្នុងករណីនេះយើងតាង $a_1=x_1/x, ...,a_n=x_n/x$ និង $b_1=y_1/x,...,b_n=y_n/y$ ដែលទាំងអស់ជាចំនួនមិនអវិជ្ជមាន។ នោះ $a_1^p+...+a_n^p=1$ និង $b_1^q+...+b_n^q=1$ ហើយវិសមភាព Holder សមមូលនិង $a_1b_1+..a_nb_n \leq 1$ ហើយយើងនិងស្រាយវាក្នុងទម្រង់សមូលនេះ។ 
តាមវិសមភាព Young យើងបាន $a_1b_1\leq \frac{a_1^p}{p}+\frac{b_1^q}{q},..., a_nb_n \leq \frac{a_n^p}{p}+\frac{b_n^q}{q}$។ បូកវិសមភាពទាំងអស់នេះយើងបាន $a_1b_1+...+a_nb_n \leq \frac{a_1^p+...a_n^p}{p}+\frac{b_1^q+...+b_n^q}{q}=1/p+1/q=1$ ព្រោះ $a_1^p+...+a_n^p=1, b_1^q+...+b_n^q=1$ និង $1/p+1/q=1$ ។
\end{proof}
{\kml វិសមភាព Mikowski} ៖ គេមាន $x_1, x_2,...,x_n$ និង $y_1,y_2,...,y_n$ ជាចំនួនពិតមិនអវិជ្ជាមាន និង $p\geq 1$ នោះ ${((x_1+y_1)^p+...+(x_n+y_n)^p)}^{1/p} \leq {(x_1^p+...+x_n^p)}^{1/p}+{(y_1^p+...+y_n^p)}^{1/p}$។

\begin{proof}
តាង $q$ ជាចំនួនវិជ្ជាមានដែល $1/p+1/q=1$ នោះ $(p-1)q=p$។ អនុវត្តវិសមភាព Holder យើងបាន
\begin{equation}
(x_1+y_1)^p+...+(x_n+y_n)^p=
(x_1+y_1)(x_1+y_1)^{p-1}+...+(x_n+y_n)(x_n+y_n)^{p-1}

= \{x_1(x_1+y_1)^{p-1}+y_1(x_1+y_1)^{p-1}\}+...+\{x_n(x_n+y_n)^{p-1}+y_n(x_n+y_n)^{p-1}\}

\leq {(x_1^p+...+x_n^p)}^{1/p}{((x_1+y_1)^{q(p-1)}+...+(x_n+y_n)^{q(p-1)})}^{1/q} 

+{(y_1^p+...+y_n^p)}^{1/p}{((x_1+y_1)^{q(p-1)}+...+(x_n+y_n)^{q(p-1)})}^{1/q}

\leq {(x_1^p+...+x_n^p)}^{1/p}{((x_1+y_1)^p+...+(x_n+y_n)^p)}^{1/q} 

+{(y_1^p+...+y_n^p)}^{1/p}{((x_1+y_1)^p+...+(x_n+y_n)^p)}^{1/q}
\end{equation}

សម្រួល ${((x_1+y_1)^p+...+(x_n+y_n)^p)}^{1/q}$ ចេញពីអង្គសាងខាងយើងបាន 

${((x_1+y_1)^p+...+(x_n+y_n)^p)}^{1-1/q} \leq {(x_1^p+...+x_n^p)}^{1/p}+{(y_1+...+y_n)}^{1/p}$ តែដោយ $1-1/q=1/p$ នោះយើងបាន

${((x_1+y_1)^p+...+(x_n+y_n)^p)}^{1/p}\leq {(x_1^p+...+x_n^p)}^{1/p}+{(y_1+...+y_n)}^{1/p}$ ។ 
\end{proof}
នៅពេលនេះយើងអាចស្រាយវិសមភាពត្រីកោណនៅក្នុងលំហរ $l^p$ បានហើយ។ យល $x=(x_1,x_2,...), y=(y_1,y_2,...)$ ជាធាតុក្នុង $l^p$។ បើ $n$ ជាចំនួនគត់ណាមួយនោះ នោះយើងបាន 
$(|x_1+y_1|^p+..+|x_n+y_n|^p)^{1/p} \leq (|x_1|^p+...+|x_n|^p)^{1/p}+(|y_1|^p+...+|y_n|^p)^{1/p} \leq (|x_1|^p+|x_2|^p+...)^{1/p}+(|y_1|^p+|y_2|^p+...)^{1/p} $ ពេលឲ្យ n ខិតទៅះអន្តនយើងបាន $(|x_1+y_1|^p+|x_n+y_n|^p+...)^{1/p} \leq (|x_1|^p+|x_2|^p+...)^{1/p}+(|y_1|^p+|y_2|^p+...)^{1/p} $ ដែលនេះគឺ $\|x+y\|_{l^p} \leq \|x\|_{l^P}+\|y\|_{l^p}$។
\\

{\kml លំហ $l^p$ ជាលំហរ Banach}
យើងបានសិក្សាខាងលើថា $l^p$ ជាលំហរវុិចទ័រ ដែលមានណមមួយ។ យើងនិងស្រាយថានៅក្នុងណមនេះ$ l^p$ មានលក្ខណៈពេញដែលថាគ្រប់ស្វុីត Cauchy នៅក្នុង $l^p$ ជាស្វុីតរួម។ 

តាង$ \{a_n\}$ ជាស្វុីត Cauchy នៅក្នុង$ l^p$។ យើងតាង $a_n=(a_n^1,a_n^2,....)$ ចំពោះគ្រប់ $n\in \mathbb{N}$។ បើ $\varepsilon >0$ នោះមាន $N$ ដែល $\|a_n-a_m\| \leq  \sqrt[p]{\varepsilon} $ ចំពោះ $n,m \leq N$ (ព្រោះ $\{a_n\}$ ជាស្វុីត Cauchy)។ ដូចនេះយើងបាន
${\sum_{k=1}^{\infty}{|a_n^k-a_m^k|^p}^{1/p}} \leq \sqrt[p]{\varepsilon}$។ នោះ $|a_n^k-a_m^k|\leq \varepsilon$ ចំពោះគ្រប់ $n,m \leq N$។ នេះមានន័យថា$ \{a_n^k\}_{n=1}^{\infty} $ជាស្វុីត Cauchy នៅក្នុង $\mathbb{R}$ ចំពោះគ្រប់ $k \in \mathbb{N}$ នោះស្វុីនេះជាស្វុីតរួម ហើយយើងតាង $a^k$ ជាលីមីតរបស់វាចំពោះគ្រប់ k និមួយៗ។ យើងឃើញថា កំប៉ូសង់របស់ស្វុីត $a_n$ សុទ្ធតែជាស្វុីតរួម។ យើងតាង $a=(a^1,a^2,...)$ ជាស្វុីតដែលតួរបស់វាមួយនិមួយៗគឺជាលីមីតរបស់តួររបស់ស្វុីត $(a_n)$។ យើងដឹងស្រាយថា a ជាធាតុមួយរបស់ $l^p$ និង ស្រាយថា $(a_n)$ រួមទៅរក a នៅក្នុង$ l^p$។ 

ដំបូងយើងឃើញថា ដោយ $(a_n)$ ជាស្វុីតកូសុីនៅក្នុង$ l^p$ នោះ បើ $\varepsilon >0$ យើងអាចរក $N \in \mathbb{N}$ ដែលធ្វើអោយ  $\sum_{k=1}^{\infty}{|a_n^k-a_m^k|^p} ={\|a_n-a_m\|_{l^p} }^p < {\varepsilon}^p$។ យក n ខិតទៅអន្តននោះយើងបាន $\sum_{k=1}^{\infty} {|a^k-a_m^k|^p} < {\varepsilon}^p$ ឬយើងអាចសរសេរបានថារ $\|a-a_m\|_{l^p} < \varepsilon$ ចំពោះគ្រប់ $m > N$ នេះមានន័យថា $(a_n)$ រួមទៅរក a នៅក្នុង $l^p$ ណម។ ម៉្យាងទៀតតាមវិសមភាពត្រីកោណ យើងបាន $\|a\|_{l^p} < \|a_m\|_{l^p} +\|a-a_m\|_{l^p}< \|a_m\|_{l^p}+\varepsilon < \infty$។ នេះមានន័យថា $a \in l^p$។
\\

ការសិក្សារបស់យើងមកដល់ត្រឹមនេះបានបង្ហាញថា មិនមែនត្រឹមតែលំហវិមាត្ររាប់អស់នោះទេដែលមានលក្ខណៈដែលយើងអាចហៅថាវុីចទ័រ លំហរវិមាត្រអន្តរក៏មានលក្ខណនេះដែរ។ ជាងនេះទៅទៀតយើងក៏អាចមានសញ្ញាណសម្រាប់វាស់ប្រវែងនៅក្នុងលំហរនេះដែរ។ រឹតតែល្អរជាងនេះ សញ្ញាណប្រវែងនៅក្នុងលំហនេះ ក៏មានលក្ខណពេញដូចនៅក្នុងលំហរវិមាត្ររាប់អស់ដែរ។ ដែលនេះជារបស់ចាំចាច់ប្រសិនបើយើងចង់ស្រាយទ្រឹស្តីបទដែលមាននៅក្នុងលំហរវិមាត្ររាប់អស់។ មុខវិទ្យា Calculus នៅក្នុងលំហរ Banach ជាមុខវិជ្ជាមួយដែលសិក្សាអំពីសញ្ញាណអាំងតេក្រាល និងដេរីវែនៅក្នុងលំហរទូទៅ ដែលឆ្ងាយពីលំហរវិមាត្ររាប់អស់ដែលយើងស្គាល់។ មុខវិទ្យានេះសម្ផស្សស្រស់ត្រកាល និងមានកាអនុវត្តច្រើនជាពិសេសនៅក្នុងង្រឹស្តីសមីកាឌឺផេរ៉ង់ស្យែល។ 
 \end{document}   
